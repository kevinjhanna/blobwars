\documentclass[10pt,a4paper,notitlepage,draft]{article}
\usepackage[utf8]{inputenc}
\usepackage[spanish]{babel}

\usepackage{amsmath}
\usepackage{amsfonts}
\usepackage{amssymb}
\usepackage{graphicx}
\usepackage{authblk}
\usepackage{makeidx}

\usepackage[utf8]{inputenc}
\renewcommand\Authand{ y }
\author[1]{\rm Kevin J. Hanna}
\author[2]{\rm Pablo A. Costesich}
\affil[ ]{Alumnos de Ingeniería Informática}
\affil[ ]{Instituto Tecnológico de Buenos Aires}
\affil[ ]{Av. Madero 399, C.A.B.A., Argentina}
\affil[1]{\textit {khanna@alu.itba.edu.ar}}
\affil[2]{\textit {pcostesi@ieee.org}}


\newenvironment{definition}[1][Definición]{\begin{trivlist}
\item[\hskip \labelsep {\bfseries #1}]}{\end{trivlist}}


%\email{khanna@alu.itba.edu.ar, pcostesi@ieee.org}

\title{Estrategia de resolución Minimax para Juegos de Suma Cero e Implementación para el juego \textit {BlobWars}}
\makeindex

\begin{document}
\maketitle
\begin{abstract}
Se presenta una introducción a la Teoría de Juegos y Juegos de Suma Cero. Estos son juegos de dos jugadores con turnos y un balance entre sus ganancias y pérdidas. La estrategia de resolución analizada en este informe es Minimax, un algoritmo de búsqueda en profundidad limitada, junto con la heurística de Poda Alfa-Beta. Se detalla la implementación y resultados prácticos.
\end{abstract}

\section{Introducción}
El Trabajo Práctico Especial de la cátedra de Estructuras de Datos y Algoritmos para el segundo cuatrimestre de 2012 propone la resolución del juego \textit{BlobWars} mediante el algoritmo Minimax, con y sin Poda Alfa-Beta, en el lenguaje de programación Java. La implementación del mismo debe contar con un modo visual y de lectura de tableros por archivo.


\subsection{BlobWars}
\textit{BlobWars} es un juego de dos jugadores por turnos en un tablero de 8x8. El objetivo es lograr la mayor cantidad de piezas sobre el tablero cuando alguno de los jugadores no pueda realizar nuevos movimientos.

Los posibles movimientos de una mancha (\textit{blob}) son siempre hacia casilleros vacíos y cumplen estas reglas:
\begin{itemize}
\item Los movimientos a distancia 1 mantienen la mancha de origen y generan una nueva del mismo color en el casillero de destino.
\item Los movimientos a distancia 2 desplazan la mancha al casillero de destino.
\item No pueden realizarse movimientos que no sean de distancia 1 o 2.
\end{itemize}

\begin{definition}
El conjunto $Manchas$ es equivalente a $\{-1, 0, 1\}$.
\end{definition}

\begin{definition}Un tablero es una matriz de $Manchas^{8x8}$.\end{definition}

\begin{definition}
Un punto es un par ordenado $\left\langle x, y\right\rangle \in Point$, donde $Point = Side \times Side$ y $Side = \{1, 2, 3, 4, 5, 6, 7, 8\}$.
\end{definition}

\begin{definition}
Se llama $Movements$ al siguiente conjunto:
\begin{equation}
Movements = Point \times Point
\end{equation}
\end{definition}

\begin{definition}
Un movimiento es un par ordenado $\left\langle inicio, destino\right\rangle \in Movements$.

Notación: $inicio \rightarrow destino$.
\end{definition}

\begin{definition}
Un movimiento válido para un jugador $p$ es un movimiento tal que el casillero de destino $d$ se encuentra libre (su valor es $0$) y el de inicio $s$ pertenece al jugador $p$ (su valor es $p$).

Notación: $v_{s \rightarrow d}(p)$
\end{definition}

\begin{definition}
Un casillero es el valor del tablero en un punto $p$.
\end{definition}

\begin{definition}
El casillero vacío es equivalente al valor $0$ en el tablero para el punto $p$.
\end{definition}

\begin{definition}
Una mancha es uno de los tres posibles estados $\in Manchas$ de una celda en un punto del tablero:
\begin{itemize}
\item[-1] \emph{Humano}: representado por la letra \textit{H}.
\item[0] \emph{Vacío}: representado por el espacio en blanco.
\item[1] \emph{Computadora}: representado por la letra \textit{C}.
\end{itemize}
\end{definition}

\begin{definition}
Se define distancia como:
\begin{equation}
    distancia(a, b) = \max(|a[x] - b[x]|, |a[y] - b[y]|)
\end{equation}

Donde $a$ y $b$ son los puntos en cuestión (tuplas de dos componentes $\mathbb{N}_{0}$), y $a[x]$ significa la componente $x$ de $a$.
\end{definition}

Adicionalmente, al final de cada movimiento la mancha en el destino \textit{infecta} a las vecinas según estas reglas:
\begin{itemize}
\item Si el casillero no se encuentra ocupado entonces se deja libre (no se altera).
\item Si el casillero se encuentra ocupado, la mancha presente en éste cambia al color de la mancha que infecta.
\end{itemize}

\begin{definition}
Aplicar un movimiento $m$ para un jugador $p$ a un tablero $b$ es aplicar las reglas de movimiento seguido de \textit{infectar} el casillero de destino. Se obtiene como resultado un tablero $b'$
\end{definition}

\begin{definition}
Un \textit{Tablero Terminal} es un tablero tal que para algún jugador no existen movimientos válidos para ninguna de sus manchas.
\begin{equation}
terminal(b) \Leftrightarrow \nexists m \in v_{s \rightarrow d}(p) \forall s, d \in Movements
\end{equation}
Donde $b$ es el tablero, $m$ un movimiento válido, $p$ un jugador y $s, d$ un movimiento.
\end{definition}

El juego termina cuando el tablero se completa o cuando el siguiente jugador no puede realizar un movimiento válido. Vale destacar que la primera condición implica la segunda.

\begin{definition}
Un turno es aplicar un movimiento válido $m$ para un jugador $p$ a un tablero $b$ y retornar el tablero resultante $b'$:
\begin{equation}
b' = turno(b, p, m)
\end{equation}
\end{definition}

\begin{definition}
Un \textit{ply} es un grupo de dos turnos que resulta de la aplicación recursiva de éstos para dos movimientos de jugadores alternados $m_p$ y $m'_{p'}$. En otras palabras: 
\begin{equation}
ply(b, m_p, m_{p'}) = turno(turno(b, p, m_p), p', m_{p'})
\end{equation}
\end{definition}

\begin{definition}
Una partida es una serie finita $G$ de $n$ \textit{plies} tal que aplicar el último movimiento ($m^{n}_{s \rightarrow p}$) al tablero $b^{n - 1}$ resulta en un tablero terminal $b^{n}$; siendo $b^{n}$ el único tablero terminal en $G$.
\end{definition}

\section{Teoría de Juegos}
La Teoría de Juegos es un campo de la matemática que estudia mediante modelos sistemas definidos por reglas de acción con incentivos (juegos), y proporciona estrategias para llevar a cabo procesos de decisión. Es un campo de estudio estrechamente ligado a las ciencias del comportamiento y las ciencias que las utilizan, como ser: economía, biología, psicología, antropología y ciencias políticas. \cite[p. 1]{bbs}

El análisis de estos modelos implica una característica diferencial con respecto a otros campos como la Teoría de la Decisión en cuanto a que los actores en la Teoría de Juegos no tienen costos e incentivos prefijados, sino que varían con las acciones de sus oponentes. También asume que cada jugador elige según un criterio de \textit{fitness} o ventaja sobre su situación y no mediante un set de reglas independiente de la misma.

\begin{definition}
\textit{Fitness} es una función de probabilidad que toma un estado o conjunto de estímulos y devuelve una valoración escalar. 
\end{definition}

Mediante el ingreso de estímulos a una constelación de entradas sensoriales, el actor racional genera una distribución estadística de probabilidad sobre distintos resultados con un \textit{fitness} asociado. \cite[p. 4]{bbs} 

Los actores racionales escogen el resultado con mejor \textit{fitness} para un entorno determinado. Por lo tanto, si un actor es presentado con una opción $A$ de menor \textit{fitness} que $B$ escogerá $B$, y si es presentado con dicho $B$ de menor \textit{fitness} que $C$ entonces escogerá $C$ por sobre $A$ y $B$. Esto es llamado \textit{Consistencia de Elección}. \cite[p. 4]{bbs}

Este modelo se llama \textit{BCP} (\textit{beliefs, preferences, and constraints} \cite[p. 3]{bbs}), e implica que distintas decisiones afectan a los resultados de las valuaciones de \textit{fitness} con distinto grado de probabilidad. La Teoría de Juegos amplía el modelo para incorporar varios jugadores, y llama Estrategias a las \textit{BCP}. Los jugadores cuentan con el conocimiento de las reglas del juego, la naturaleza de sus contrincantes y estrategias disponibles. Para cada combinación de decisiones, para cada jugador, el juego le asigna una distribución de \textit{individual payoffs} (ganancias individuales). Con esto, la Teoría de Juegos logra predecir el comportamiento de cada jugador (al asumir que maximizará sus ganancias, al ser un actor racional). \cite[p. 8]{bbs}

Cabe destacar que el modelo no es aplicable a todo tipo de modelo válido, ya que en la práctica puede ocurrir que los actores no sean enteramente racionales, desconozcan parte de la información en un estado determinado o no sean consistentes en sus decisiones (por ejemplo, un animal que se vuelve adicto a un alimento determinado). \cite[p. 9 - 11, ]{bbs}

El problema más grave es que los actores racionales no respeten el Equilibrio de Nash. \cite{ram}

\subsection{Equilibrio de Nash}
El Equilibrio de Nash es un concepto de solución para un juego no cooperativo de dos o más jugadores, en donde todos ellos tienen conocimiento de las estrategias de equilibrio de los otros.

Si cada jugador ha elegido una estrategia y ningún jugador puede beneficiarse de un cambio de la misma cuando sus oponentes mantienen las de ellos, entonces se afirma que dicho conjunto de estrategias y sus ganancias individuales constituye un Equilibrio de Nash. \cite{osb}

Si un jugador $A$ tiene una estrategia dominante $S_A$ entonces existe un Equilibrio de Nash en donde $A$ juega $S_A$. En el caso de ser dos jugadores $A$ y $B$, existe un Equilibrio en donde $A$ juega $S_A$ y $B$ juega la mejor estrategia en respuesta a $S_A$ (puede no ser única). Si $S_A$ es una estrategia estrictamente dominante, $A$ juega $S_A$ en todo Equilibrio de Nash. Si ambos $A$ y $B$ tienen estrategias estricamente dominantes, existe un único Equilibrio de Nash en donde ambos jugadores juegan sus estrategias dominantes respectivas. 

\subsection{Juegos de Suma Cero}

Los juegos de suma cero responden a la premisa básica de que las ganancias de un jugador se equiparan a la pérdida conjunta de los otros. Asume que todos los jugadores conocen el Equilibrio de Nash de los otros.

\section{Estructuras de Datos}
\section{Algoritmos}
\subsection{Minimax (Na\"ive)}
\subsection{Minimax con Poda Alfa-Beta}
\section{Problemas encontrados}
\subsection{Uso eficiente de memoria}
  Inicialmente cuando se desarrollaron las clases del juego, se notó que consumían memoria demás.
  Eso se vió demostrado ya que el algoritmo propuesto para minimax no podía llegar hasta el nivel 5.
  A continuación se detallan los puntos críticos en donde se tuvo que repensar la implementación.

  \subsection{Clase Board}
    En un principio, la forma utilizada para representar el board fue utilizando objetos Tile, del cual heredaban EmptyTile y BlobTile.
    Es decir, que la cantidad de referencias de cada tablero era siempre de 64 $(8 * 8)$.  A la vez, cada BlobTile contenía una referencia a un Player.

    Para iterar facilmente por los tiles de un Player y obtener sus jugadas posibles se contó con un $HashMap<Player, ArrayList<Point>>$, y un método para acceder a esos  puntos en donde se encontraban los tiles de cada jugador.

    Para hacer uso más eficiente de la memoria, se decidió implementar el Board cómo un array primitivo de Player, y un método para convertir de Point al indice de ese array.
    En este caso se hizo un cambio que consume menos memoria al haber menos referencias y objetos en tiempo de ejecucón, aunque es cierto que esa solución hace que haya que calcular más en tiempo de ejecución.

    El cambio más radical en cuanto a cantidad de memoria consumida, se logró implementando un iterador de Board para ser utilizado al generar los tableros con
    las jugadas posibles a partir de cierto Board.

\section{Comparaciones y pruebas}
\section{Implementación}
\section{Conclusiones}
\section{Anexo}
\subsection{Implementación alternativa del núcleo del Solver}
\begin{thebibliography}{9}
  % type bibliography here
  \bibitem[BBS]{bbs} \emph{A framework for the unification of the behavioral sciences}, Herbert Gintis, Behavioral and Brain Sciences (2007) 30:1-61
  
  \bibitem[RAM]{ram} Chen, H.-C., Friedman, J. W. \& Thisse, J.-F. (1997) Boundedly rational Nash
equilibrium: A probablistic choice approach. Games and Economic Behavior
18:32– 54.

  \bibitem[OSB]{osb} Osborne, Martin J., and Ariel Rubinstein. A Course in Game Theory. Cambridge, MA: MIT, 1994. Print.

\end{thebibliography}
\end{document}
